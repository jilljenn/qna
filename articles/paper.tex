\documentclass{article}
\usepackage[utf8]{inputenc}
\usepackage[T1]{fontenc}
\title{Adaptive tests using q-matrices}
\author{Jill-Jênn Vie \and TMTC}

\newtheorem{jb}{Nota Jibe}

\begin{document}
\maketitle

\begin{abstract}
To our knowledge, this is the first use of q-matrices for adaptive tests.
\end{abstract}

\begin{jb}
Deadline le 15 juin.
\end{jb}

\section{Introduction}



\section{Definition}

A user is modeled by a binary vector, called \emph{state}, representing her master of the different skills.

A $Q$-matrix represents, for each question, the different skills required to answer the question. $Q_{ij}$ is equal to 1 if the skill $j$ is required to solve the question $i$, 0 otherwise.

Recent work tend to infer such q-matrices from student data. Based on this structure, it is possible to refine an estimation of a newcomer's skills, according to her answers. Indeed, according to the presupposed state of a newcomer's current skills, it is interesting to pick the question that will bring the most information about her true skills.

\begin{jb}
Desmarais a écrit beaucoup de papiers sur le sujet et m'a confirmé que les q-matrices n'avaient jamais été utilisées à des fins de tests.
\end{jb}

\section{Our contribution}

This paper tends to address three cases:
\begin{itemize}
\item every student answers according to its true skills;
\item every student may fail one question whereas she has the correct skills or answer correctly while she hasn't the required skills, according to a \emph{slip} parameter;
\item an new question is added to the test, the skills required to answer to it being unknown.
\end{itemize}

\section{Test framework}

\section{Results}

\end{document}