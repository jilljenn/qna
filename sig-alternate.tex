% This is "sig-alternate.tex" V2.0 May 2012
% This file should be compiled with V2.5 of "sig-alternate.cls" May 2012
%
% This example file demonstrates the use of the 'sig-alternate.cls'
% V2.5 LaTeX2e document class file. It is for those submitting
% articles to ACM Conference Proceedings WHO DO NOT WISH TO
% STRICTLY ADHERE TO THE SIGS (PUBS-BOARD-ENDORSED) STYLE.
% The 'sig-alternate.cls' file will produce a similar-looking,
% albeit, 'tighter' paper resulting in, invariably, fewer pages.
%
% ----------------------------------------------------------------------------------------------------------------
% This .tex file (and associated .cls V2.5) produces:
%       1) The Permission Statement
%       2) The Conference (location) Info information
%       3) The Copyright Line with ACM data
%       4) NO page numbers
%
% as against the acm_proc_article-sp.cls file which
% DOES NOT produce 1) thru' 3) above.
%
% Using 'sig-alternate.cls' you have control, however, from within
% the source .tex file, over both the CopyrightYear
% (defaulted to 200X) and the ACM Copyright Data
% (defaulted to X-XXXXX-XX-X/XX/XX).
% e.g.
% \CopyrightYear{2007} will cause 2007 to appear in the copyright line.
% \crdata{0-12345-67-8/90/12} will cause 0-12345-67-8/90/12 to appear in the copyright line.
%
% ---------------------------------------------------------------------------------------------------------------
% This .tex source is an example which *does* use
% the .bib file (from which the .bbl file % is produced).
% REMEMBER HOWEVER: After having produced the .bbl file,
% and prior to final submission, you *NEED* to 'insert'
% your .bbl file into your source .tex file so as to provide
% ONE 'self-contained' source file.
%
% ================= IF YOU HAVE QUESTIONS =======================
% Questions regarding the SIGS styles, SIGS policies and
% procedures, Conferences etc. should be sent to
% Adrienne Griscti (griscti@acm.org)
%
% Technical questions _only_ to
% Gerald Murray (murray@hq.acm.org)
% ===============================================================
%
% For tracking purposes - this is V2.0 - May 2012

\documentclass{sig-alternate}
\usepackage[T1]{fontenc}
\usepackage[utf8]{inputenc}
\usepackage{algorithm,caption,algpseudocode}

\begin{document}
%
% --- Author Metadata here ---
\conferenceinfo{ASSESS}{2014 New York City, USA}
%\CopyrightYear{2007} % Allows default copyright year (20XX) to be over-ridden - IF NEED BE.
%\crdata{0-12345-67-8/90/01}  % Allows default copyright data (0-89791-88-6/97/05) to be over-ridden - IF NEED BE.
% --- End of Author Metadata ---

\title{A Novel Approach to Adaptive Testing\titlenote{(Produces the permission block, and
copyright information). For use with
SIG-ALTERNATE.CLS. Supported by ACM.}}
\subtitle{English subtitles
\titlenote{A full version of this paper is available as
\textit{Author's Guide to Preparing ACM SIG Proceedings Using
\LaTeX$2_\epsilon$\ and BibTeX} at
\texttt{www.acm.org/eaddress.htm}}}
%
% You need the command \numberofauthors to handle the 'placement
% and alignment' of the authors beneath the title.
%
% For aesthetic reasons, we recommend 'three authors at a time'
% i.e. three 'name/affiliation blocks' be placed beneath the title.
%
% NOTE: You are NOT restricted in how many 'rows' of
% "name/affiliations" may appear. We just ask that you restrict
% the number of 'columns' to three.
%
% Because of the available 'opening page real-estate'
% we ask you to refrain from putting more than six authors
% (two rows with three columns) beneath the article title.
% More than six makes the first-page appear very cluttered indeed.
%
% Use the \alignauthor commands to handle the names
% and affiliations for an 'aesthetic maximum' of six authors.
% Add names, affiliations, addresses for
% the seventh etc. author(s) as the argument for the
% \additionalauthors command.
% These 'additional authors' will be output/set for you
% without further effort on your part as the last section in
% the body of your article BEFORE References or any Appendices.

\numberofauthors{8} %  in this sample file, there are a *total*
% of EIGHT authors. SIX appear on the 'first-page' (for formatting
% reasons) and the remaining two appear in the \additionalauthors section.
%
\author{
% You can go ahead and credit any number of authors here,
% e.g. one 'row of three' or two rows (consisting of one row of three
% and a second row of one, two or three).
%
% The command \alignauthor (no curly braces needed) should
% precede each author name, affiliation/snail-mail address and
% e-mail address. Additionally, tag each line of
% affiliation/address with \affaddr, and tag the
% e-mail address with \email.
%
% 1st. author
\alignauthor Jill-Jênn Vie\titlenote{Blah blah blah.}\\
       \affaddr{ENS Cachan -- Bât. Cournot}\\
       \affaddr{61 av. du Président Wilson}\\
       \affaddr{94235 Cachan, France}\\
       \email{vie@jill-jenn.net}
% 2nd. author
\alignauthor
Fabrice Popineau\titlenote{Blah blah blah.}\\
       \affaddr{Supélec -- Dép. informatique}\\
       \affaddr{3 rue Joliot Curie}\\
       \affaddr{91192 Gif-sur-Yvette, France}\\
       \email{fabrice.popineau@supelec.fr}
% 3rd. author
\alignauthor
Yolaine Bourda\titlenote{Blah blah blah.}\\
       \affaddr{Supélec -- Dép. informatique}\\
       \affaddr{3 rue Joliot Curie}\\
       \affaddr{91192 Gif-sur-Yvette, France}\\
       \email{yolaine.bourda@supelec.fr}
}
% There's nothing stopping you putting the seventh, eighth, etc.
% author on the opening page (as the 'third row') but we ask,
% for aesthetic reasons that you place these 'additional authors'
% in the \additional authors block, viz.
% \additionalauthors{Additional authors: John Smith (The Th{\o}rv{\"a}ld Group,
% email: {\texttt{jsmith@affiliation.org}}) and Julius P.~Kumquat
% (The Kumquat Consortium, email: {\texttt{jpkumquat@consortium.net}}).}
\date{May 26, 2014}
% Just remember to make sure that the TOTAL number of authors
% is the number that will appear on the first page PLUS the
% number that will appear in the \additionalauthors section.

\maketitle
\begin{abstract}
Q-matrices To our knowledge, this is the first use of q-matrices for adaptive tests.
\end{abstract}

% A category with the (minimum) three required fields
\category{}{Student assessment}{Adaptive testing}
%A category including the fourth, optional field follows...
\category{}{Active learning}{Generalized binary search}

\terms{Adaptive tests, q-matrices}

\keywords{Adaptive assessment}

\section{Introduction}

Let us consider a test composed of multiple-choice questions. If we already have a gigantic database of who answered what, it is natural to wonder which questions bring the most information about an examinee, of if there is a ``best sequence'' according to which the questions should be asked. % process in an interview

Indeed, if the examinee behaves typically, a subset of results may be enough to guess how she will perform on the rest of the questions, using common response patterns.

This is the idea behind item response theory. Users are modeled by a latent parameter $\theta$ called \emph{ability}, which enables the estimation of probability response items. % flou
Throughout the test, the question that will bring the most information about $\theta$ according to some criteria is asked to the examinee. Thus, a confidence interval around $\theta$ is devised and resserré. % TODO

A q-matrix is a model link between competences and results. skills required to answer

Recent work tend to infer such q-matrices from student data. Based on this structure, it is possible to refine an estimation of a newcomer's skills, according to her answers. Indeed, according to the presupposed state of a newcomer's current skills, it is interesting to pick the question that will bring the most information about her true skills.

\subsection{Our contribution}

This paper compares the IRT approach to the Q-matrix approach.

\subsection{Outline}

We will first present both models, explain how they allow us to build a test satisfying our needs, then how they behave differently on simulated data and real data.

\section{Preliminaries}

\subsection{Models}

A user is modeled by a binary vector, called \emph{state}, representing her master of the different skills.

The student data matrix.

\paragraph{Item response theory} Well-known method based on a Rasch model
\[ \Pr{lol} = \frac1{e^{dis(\theta - d)}} \]

\paragraph{Q-matrices} A $Q$-matrix represents, for each question, the different skills required to answer the question. $Q_{ij}$ is equal to 1 if the skill $j$ is required to solve the question $i$, 0 otherwise. For example. Required.

\subsection{Generalized Binary Search}

Eliminate hypotheses using bayesian thing.

\section{Test framework}

\begin{algorithm}
\caption*{\textbf{Adaptive testing using IRT}}
\begin{algorithmic}
\Procedure{Test}{}
\While{$entropy(\pi) > threshold$}
	\State Pick a question
	\State Update $\pi$
\EndWhile
\EndProcedure
\end{algorithmic}
\end{algorithm}

\begin{algorithm}
\caption*{\textbf{Adaptive testing using Q-matrix}}
\begin{algorithmic}
\Procedure{Test}{}
\While{$entropy(\pi) > threshold$}
	\State Pick a question
	\State Update $\pi$
\EndWhile
\EndProcedure
\end{algorithmic}
\end{algorithm}

We used cross-validation to compare those scenarios.

\section{Results}

\subsection{Simulated data}

OMG. Super graphe.

\subsection{True data}

LSAT.

\section{Conclusions}

Outperforms on both simulated data and real data.

\section{Acknowledgments}

We wish to thank JBG for his valuable comments.

% The following two commands are all you need in the
% initial runs of your .tex file to
% produce the bibliography for the citations in your paper.
\bibliographystyle{abbrv}
\bibliography{sigproc}  % sigproc.bib is the name of the Bibliography in this case
% You must have a proper ".bib" file
%  and remember to run:
% latex bibtex latex latex
% to resolve all references
%
% ACM needs 'a single self-contained file'!
%
%APPENDICES are optional
%\balancecolumns
\appendix
%Appendix A
\section{If I have something more to say}

%\balancecolumns % GM June 2007
% That's all folks!
\end{document}
